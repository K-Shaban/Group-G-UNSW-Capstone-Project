\documentclass[mstat,12pt]{unswthesis}


\newlength{\cslhangindent}
\setlength{\cslhangindent}{1.5em}
\newenvironment{CSLReferences}%
  {}%
  {\par}

%%%%%%%%%%%%%%%%%%%%%%%%%%%%%%%%%%%%%%%%%%%%%%%%%%%%%%%%%%%%%%%%%%
% 
% OK...Now we get to some actual input.  The first part sets up
% the title etc that will appear on the front page
%
%%%%%%%%%%%%%%%%%%%%%%%%%%%%%%%%%%%%%%%%%%%%%%%%%%%%%%%%%%%%%%%%%

\title{Group Project Plan by Group G\\[0.5cm]A Data Science Approach to study the impact of COVID-19 Lockdowns and Work-From-Home Trend on Electricity Demand in New South Wales}

\authornameonly{Kourosh Shaban (z5016011), Kin Ieng Cheang (z5003196), \\ Alexander Hodges (z5411346), Zining Wang (z5401010)}

\author{\Authornameonly}

\copyrightfalse
\figurespagefalse
\tablespagefalse

%%%%%%%%%%%%%%%%%%%%%%%%%%%%%%%%%%%%%%%%%%%%%%%%%%%%%%%%%%%%%%%%%
%
%  And now the document begins
%  The \beforepreface and \afterpreface commands puts the
%  contents page etc in
%
%%%%%%%%%%%%%%%%%%%%%%%%%%%%%%%%%%%%%%%%%%%%%%%%%%%%%%%%%%%%%%%%%%


\input{header.tex}

\begin{document}

\beforepreface



%\afterpage{\blankpage}


\afterpreface





%%%%%%%%%%%%%%%%%%%%%%%%%%%%%%%%%%%%%%%%%%%%%%%%%%%%%%%%%%%%%%%%%%
%
% Now we can start on the first chapter
% Within chapters we have sections, subsections and so forth
%
%%%%%%%%%%%%%%%%%%%%%%%%%%%%%%%%%%%%%%%%%%%%%%%%%%%%%%%%%%%%%%%%%%



%%%%%%%%%%%%%%%%%%%%%%%%%%%%%%%%%%%%%

%\afterpage{\blankpage}


\setcounter{chapter}{1}
\renewcommand\thesection{\arabic{section}}

\hypertarget{introduction-and-motivation}{%
\section{Introduction and
Motivation}\label{introduction-and-motivation}}

The Covid-19 pandemic resulted in a myriad of changes in the behaviour
of individuals, businesses, and global economy. In the context of energy
forecasting, the lockdowns due to the pandemic and the advent of working
from home are interesting phenomena that have left notable impressions on the demand and supply of energy. Additionally, the growth of the digital economy was exacerbated as a result of the events of the pandemic and this in itself has transformed consumer and business energy use. Recent studies on global and Australian energy demand have indicated:
\begin{enumerate}
  \item Overall demand has reduced whilst residential demand has increased
  \item The pattern of demand has changed during the day and during the week
\end{enumerate}

\bigskip

\noindent Our goal in this analysis is to isolate the effects of lockdowns, and working from home from the pre-Covid trend in the present energy demand curve. In doing so we can generate valuable insights on how the persistent shift to working from home will continue to effect energy demand in the future. To achieve this goal we will begin by determining a high-accuracy pre-Covid 5 minute energy demand forecasting model. To obtain this we will compare Multiple Linear Regression models, Random Forests, and Neural Networks using the weather and total energy demand data. This model will be applied to the Covid-to-present portion of the data and the output will be compared with the actual values to identify the impression of the pandemic and working from home on the pre-Covid trend.

\hypertarget{brief-literature-review}{%
\section{Brief Literature Review}\label{brief-literature-review}}

To prevent the spread of Covid-19, governments worldwide implemented a range of measures including lockdowns, which has significantly influenced people’s living style. This has led to unprecedented changes in electricity demand patterns, prompting studies in a lot of countries. For example, during lockdowns, Spain experienced a 13.49\% decrease in electricity demand, with marked declines on workdays and altered daily usage patterns (Santiago et al. 2021). Ontario in Canada also witnessed reduced demand and shifted weekly usage pattern with a flattening daily demand curve since the pandemic (Abu-Rayash \& Dincer, 2020). While other countries such as China, Italy and India also found decreased demand overall, Krarti and Aldubyan (2021) used weather adjusted time series data and revealed as much as a 30\% increase of consumption in houses during lockdowns. 

\bigskip

Australia also recorded a decrease of 6.7\% in electricity demand overall a month after the implementation of Covid-19 measures (Farrow 2020). In particular, several studies examined the implication on commercial and residential demand in Victoria using weather adjusted comparison and regression models, showing similar results to other countries mentioned above (Farrow 2020; Wu et al. 2023). However, in New South Wales (NSW), it is challenging to separate the effect of residential and commercial demand due to the lack of smart metre installations (Snow et al. 2020), resulting in a lack of study in the effect of stay-at-home measure in NSW. Moreover, the adoption of hybrid working style continues to affect the electricity demand in Australian households even after the pandemic, resulting in a need of study in this area. When AEMO analysed the impact of Covid-19 on electricity demand, a “frozen in time” concept was adopted in their forecasting model which inputted pre-Covid-19 conditions and forecasted the demand for comparison with the actual demand during Covid-19 (AEMO 2020). A similar approach can be applied to study the trend of working from home on electricity demand in NSW post-Covid-19 and solve the challenge of isolating residential demand, assuming all commercial activities have resumed. 


\hypertarget{methods-software-and-data-description}{%
\section{Methods, Software and Data
Description}\label{methods-software-and-data-description}}

Throughout this project, we will use data processing, storage, analysis, and various modelling techniques to predict ongoing energy demand in NSW. We will start by considering total energy demand data and forecast data, which will be processed and cleaned by removing outliers using the IQR elimination method. The data will then be normalised before exploring the best methods to incorporate it into predictive signals. The identification of various features will be important for developing potential uses for each of the individual signals. We will identify features using the Augmented Dickey-Fuller test to review autoregressive relationships and minimise them if necessary by analysing change in signal space instead of time horizon. Cross-correlation and other time series analysis tools will be used to identify correlations and select time frames for acting on signals. We will review any signal changes during the specific time frame of 2020-2022 and analyse residuals to isolate and incorporate additional load from work-at-home activities. Our goal is to experiment with a variety of models such as Multiple Linear Regression models, Random Forests, and Neural Networks to predict energy demand at time T+1 based on discovered signals, including changes in load during the pandemic.

\bigskip

Python will be our main analysis tool with the report being constructed in Jupyter. Jupyter notebook provides an efficient system for code isolation and simultaneous work on multiple sections. It will also enable parallelisation of signal analysis throughout the project and enable model expansion. Python is preferred due to our experience with the language and its rich package ecosystem for analysis and modeling. We will primarily use Pandas and NumPy for data handling and processing, while scikit-learn library will be used for model construction, including MLR, Neural Network, and Random Forest Models with the extensions to the XGBoost Adaptive learning models. The data will be processed into training and test datasets, and converted into HDF format for efficient access and processing. Supplementary datasets will be converted to the same storage style. This modular storage system will help minimise the curse of dimensionality as more signals are introduced to the model. The initial model will be built from the base of using the temperature as a variable to predict the energy usage.

\hypertarget{activities-and-schedule}{%
\section{Activities and Schedule}\label{activities-and-schedule}}
\setlength{\parindent}{0pt} %

%\vspace{-4cm}\section{Activities and Schedule}

\textbf{Description of Activity Allocation:}
Our project has 13 main activities, five of which are done jointly by all team members and the other seven are allocated to the most suitable team members based on their strengths. For example, Kin is good at research and has flexible working hours, so she was selected as the leader of our team, responsible for making Literature Review, deciding research questions and presentation slides. The following table shows the strengths of the group members and the activities they are responsible for.
\begin{table}[htbp]
	\resizebox{1\textwidth}{!}{
	\begin{tabular}{|1|cll|}
		\hline
		\multicolumn{1}{|c|}{Tasks} & \multicolumn{1}{c|}{Members Name} & \multicolumn{1}{c|}{Members Roles} & \multicolumn{1}{c|}{Members Strengths and Experiences} \\ \hline
		Decide roles of members & \multicolumn{3}{c|}{All Members} \\ \hline
		Decide research question & \multicolumn{3}{c|}{All Members} \\ \hline
		Write project plan & \multicolumn{3}{c|}{All Members} \\ \hline
		Write report & \multicolumn{3}{c|}{All Members} \\ \hline
		Source external data & \multicolumn{3}{c|}{All Members} \\ \hline
		Record presentation & \multicolumn{3}{c|}{All Members} \\ \hline
		& \multicolumn{1}{c|}{Kin Cheang} & \multicolumn{1}{c|}{Team Leader} & Coding, Research, Mathematics \\ \cline{2-4} 
		\multirow{-2}{*}{Literature review} & \multicolumn{1}{c|}{Zining Wang} & \multicolumn{1}{c|}{Data Analyst} & Coding, Data visualization, Research \\ \hline
		& \multicolumn{1}{c|}{Alexander Hodges} & \multicolumn{1}{c|}{} & Coding, Machine learning, Mathematics \\ \cline{2-2} \cline{4-4} 
		\multirow{-2}{*}{Research methods and software} & \multicolumn{1}{c|}{Zining Wang} & \multicolumn{1}{c|}{\multirow{-2}{*}{Data Analyst}} & Coding, Data visualization, Research \\ \hline
		& \multicolumn{1}{c|}{Kourosh Shaban} & \multicolumn{1}{c|}{} & Coding, Machine learning, Mathematics \\ \cline{2-2} \cline{4-4} 
		& \multicolumn{1}{c|}{Alexander Hodges} & \multicolumn{1}{c|}{} & Coding, Machine learning, Mathematics \\ \cline{2-2} \cline{4-4} 
		\multirow{-3}{*}{Data preparation} & \multicolumn{1}{c|}{Zining Wang} & \multicolumn{1}{c|}{\multirow{-3}{*}{Data Analyst}} & Coding, Data visualization, Research \\ \hline
		& \multicolumn{1}{l|}{Kourosh Shaban} & \multicolumn{1}{l|}{} & Coding, Machine learning, Mathematics \\ \cline{2-2} \cline{4-4} 
		& \multicolumn{1}{l|}{Alexander Hodges} & \multicolumn{1}{l|}{} & Coding, Machine learning, Mathematics \\ \cline{2-2} \cline{4-4} 
		\multirow{-3}{*}{Data analysis} & \multicolumn{1}{l|}{Zining Wang} & \multicolumn{1}{l|}{\multirow{-3}{*}{Data Analyst}} & Coding, Data visualization, Research \\ \hline
		& \multicolumn{1}{l|}{Kourosh Shaban} & \multicolumn{1}{l|}{} & Coding, Machine learning, Mathematics \\ \cline{2-2} \cline{4-4} 
		& \multicolumn{1}{l|}{Alexander Hodges} & \multicolumn{1}{l|}{} & Coding, Machine learning, Mathematics \\ \cline{2-2} \cline{4-4} 
		\multirow{-3}{*}{Modelling} & \multicolumn{1}{l|}{Zining Wang} & \multicolumn{1}{l|}{\multirow{-3}{*}{Data Analyst}} & Coding, Data visualization, Research \\ \hline
		& \multicolumn{1}{l|}{Kourosh Shaban} & \multicolumn{1}{l|}{} & Coding, Machine learning, Mathematics \\ \cline{2-2} \cline{4-4} 
		& \multicolumn{1}{l|}{Alexander Hodges} & \multicolumn{1}{l|}{} & Coding, Machine learning, Mathematics \\ \cline{2-2} \cline{4-4} 
		\multirow{-3}{*}{Evaluation of model} & \multicolumn{1}{l|}{Zining Wang} & \multicolumn{1}{l|}{\multirow{-3}{*}{Data Analyst}} & Coding, Data visualization, Research \\ \hline
		Presentation slides & \multicolumn{1}{l|}{Kin Cheang} & \multicolumn{1}{l|}{Team Leader} & Coding, Research, Mathematics \\ \hline
	\end{tabular}}
\end{table}

\textbf{Setting the start and end time of each activity:}
The start time and end time of each activity are set based on the due time of each assignment in the course. To give team members enough time to review work and respond to unexpected issues, activities end one day earlier than the actual due date for each assignment. For example, the deadline for the final report is October 7, and the deadline for our group is October 6.

\bigskip

\textbf{Description of project plan and schedule (Appendix Figure 1.1):}
Project plan and schedule custom gantt charts made in excel. In order to make team members easy to use, gantt chart has only two manual functions: 
\begin{itemize}
    \item One is the name section in the upper right corner of the chart, and team members can view the progress of individual members/team members by clicking on the single/multiple name option. \item The other function is timeline cover, which allows members to drag the right side of the timeline cover to any date, with past times and completed activities in the dark side and future times and unfinished activities in the bright side.	
\end{itemize}


	
%\end{document}%closing statement-body
\bigskip

\hypertarget{references}{%
\chapter*{References}\label{references}}
\addcontentsline{toc}{chapter}{References}

\bibliographystyle{elsarticle-num}
\bibliography{references}

\hypertarget{refs}{}

\leavevmode\vadjust pre{\hypertarget{ref-Lafaye2013}{}}%
\hangindent=2em
\hangafter=1
Abu-Rayash, A. \& Dincer, I., 2020, ‘Analysis of the electricity demand trends amidst the COVID-19 coronavirus pandemic’, Energy Research \& Social Science, vol. 68, pp.101682. \hfill\break

\hangindent=2em
\hangafter=1
\noindent Australian Energy Market Operator (AEMO) 2020, COVID-19 Demand Impact in Australia, assessed 30 August 2023, $<$https://aemo.com.au/en/newsroom/news-updates/demand-impact-australia-covid19$>$. \hfill\break

\hangindent=2em
\hangafter=1
\noindent Farrow, H. 2020, Commercial down v residential up: COVID-19’s electricity impact, Energy Networks Australia, assessed 30 August 2023, $<$https://www. energynetworks.com.au/news/energy-insider/2020-energy-insider/commercial -down-v-residential-up-covid-19s-electricity-impact/\#\_edn2$>$. \hfill\break

\hangindent=2em
\hangafter=1
\noindent Krarti, M. \& Aldubyan, M. 2021, ‘Review analysis of COVID-19 impact on electricity demand for residential buildings’, Renewable and Sustainable Energy Reviews, vol. 143, pp.110888. \hfill\break

\hangindent=2em
\hangafter=1
\noindent Santiago, I., Moreno-Munoz, A., Quintero-Jiménez, P., Garcia-Torres, F. \& Gonzalez-Redondo, M.J. 2021, ‘Electricity demand during pandemic times: The case of the COVID-19 in Spain’, Energy Policy, vol. 148, pp.111964. \hfill\break

\hangindent=2em
\hangafter=1
\noindent Snow, S., Bean, R., Glencross, M. \& Horrocks, N. 2020, ‘Drivers behind residential electricity demand fluctuations due to COVID-19 restrictions’, Energies, vol. 13, no. 21, pp.5738. \hfill\break

\hangindent=2em
\hangafter=1
\noindent Wu, J., Levi, N., Araujo, R. \& Wang, Y.G. 2023, ‘An evaluation of the impact of COVID-19 lockdowns on electricity demand’, Electric Power Systems Research, vol. 216, pp.109015. \hfill\break

\bigskip

\hypertarget{appendix}{%
\chapter*{Appendix}\label{appendix}}
\addcontentsline{toc}{chapter}{Appendix}

	\begin{figure}[htbp]%opening statement-body
		\centering
		
		\includegraphics[scale=0.39,angle=270]{GanttChart.jpg} 
		\caption{Project Plan and Timeline}
	\end{figure}%closing statement-body






\end{document}

