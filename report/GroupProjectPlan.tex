\documentclass[mstat,12pt]{unswthesis}


\newlength{\cslhangindent}
\setlength{\cslhangindent}{1.5em}
\newenvironment{CSLReferences}%
  {}%
  {\par}

%%%%%%%%%%%%%%%%%%%%%%%%%%%%%%%%%%%%%%%%%%%%%%%%%%%%%%%%%%%%%%%%%%
% 
% OK...Now we get to some actual input.  The first part sets up
% the title etc that will appear on the front page
%
%%%%%%%%%%%%%%%%%%%%%%%%%%%%%%%%%%%%%%%%%%%%%%%%%%%%%%%%%%%%%%%%%

\title{Group Project Plan by Group G\\[0.5cm]A Data Science Approach to
Forecast Electricity Consumption in Australia}

\authornameonly{Kin Leng Cheng (zID), Zining Wang (zID5401010), Kourosh Shaban (z5016011), Alexander Hodges (z5411346).}

\author{\Authornameonly}

\copyrightfalse
\figurespagefalse
\tablespagefalse

%%%%%%%%%%%%%%%%%%%%%%%%%%%%%%%%%%%%%%%%%%%%%%%%%%%%%%%%%%%%%%%%%
%
%  And now the document begins
%  The \beforepreface and \afterpreface commands puts the
%  contents page etc in
%
%%%%%%%%%%%%%%%%%%%%%%%%%%%%%%%%%%%%%%%%%%%%%%%%%%%%%%%%%%%%%%%%%%


%%%%%%%%%%%%%%%%%%%%%%%%%%%%%%%%%%%%%%%%%%%%%%%%%%%%%%%%%%%%%%%%%%%%%%%
%
%  A small sample UNSW Coursework Masters thesis file.
%  Any questions to Ian Doust i.doust@unsw.edu.au and/or Gery Geenens ggeenens@unsw.edu.au
%
%%%%%%%%%%%%%%%%%%%%%%%%%%%%%%%%%%%%%%%%%%%%%%%%%%%%%%%%%%%%%%%%%%%%%%%
%
%  The first part pulls in a UNSW Thesis class file.  This one is
%  slightly nonstandard and has been set up to do a couple of
%  things automatically
%
 
%%%%%%%%%%%%%%%%%
%% Precisely one of the next four lines should be uncommented.
%% Choose the one which matches your degree, uncomment it, and comment out the other two!
%\documentclass[mfin,12pt]{unswthesis}    %%  For Master of Financial Mathematics 
%\documentclass[mmath,12pt]{unswthesis}   %%  For Master of Mathematics
%\documentclass[mstat,12pt]{unswthesis}  %%  For Master of Statistics
%%%%%%%%%%%%%%%%%



\linespread{1}
\usepackage{amsfonts}
\usepackage{amssymb}
\usepackage{amsthm}
\usepackage{latexsym,amsmath}
\usepackage{graphicx}
\usepackage{afterpage}
\usepackage[colorlinks]{hyperref}
 \hypersetup{
     colorlinks=true,
     linkcolor=blue,
     filecolor=blue,
     citecolor= black,      
     urlcolor=cyan,
     }
\usepackage{textcomp}
\usepackage{longtable}
\usepackage{booktabs}

%%%%%%%%%%%%%%%%%%%%%%%%%%%%%%%%%%%%%%%%%%%%%%%%%%%%%%%%%%%%%%%%%
%
%  The following are some simple LaTeX macros to give some
%  commonly used letters in funny fonts. You may need more or less of
%  these
%
\newcommand{\R}{\mathbb{R}}
\newcommand{\Q}{\mathbb{Q}}
\newcommand{\C}{\mathbb{C}}
\newcommand{\N}{\mathbb{N}}
\newcommand{\F}{\mathbb{F}}
\newcommand{\PP}{\mathbb{P}}
\newcommand{\T}{\mathbb{T}}
\newcommand{\Z}{\mathbb{Z}}
\newcommand{\B}{\mathfrak{B}}
\newcommand{\BB}{\mathcal{B}}
\newcommand{\M}{\mathfrak{M}}
\newcommand{\X}{\mathfrak{X}}
\newcommand{\Y}{\mathfrak{Y}}
\newcommand{\CC}{\mathcal{C}}
\newcommand{\E}{\mathbb{E}}
\newcommand{\cP}{\mathcal{P}}
\newcommand{\cS}{\mathcal{S}}
\newcommand{\A}{\mathcal{A}}
\newcommand{\ZZ}{\mathcal{Z}}
%%%%%%%%%%%%%%%%%%%%%%%%%%%%%%%%%%%%%%%%%%%%%%%%%%%%%%%%%%%%%%%%%%%%%
%
% The following are much more esoteric commands that I have left in
% so that this file still processes. Use or delete as you see fit
%
\newcommand{\bv}[1]{\mbox{BV($#1$)}}
\newcommand{\comb}[2]{\left(\!\!\!\begin{array}{c}#1\\#2\end{array}\!\!\!\right)
}
\newcommand{\Lat}{{\rm Lat}}
\newcommand{\var}{\mathop{\rm var}}
\newcommand{\Pt}{{\mathcal P}}
\def\tr(#1){{\rm trace}(#1)}
\def\Exp(#1){{\mathbb E}(#1)}
\def\Exps(#1){{\mathbb E}\sparen(#1)}
\newcommand{\floor}[1]{\left\lfloor #1 \right\rfloor}
\newcommand{\ceil}[1]{\left\lceil #1 \right\rceil}
\newcommand{\hatt}[1]{\widehat #1}
\newcommand{\modeq}[3]{#1 \equiv #2 \,(\text{mod}\, #3)}
\newcommand{\rmod}{\,\mathrm{mod}\,}
\newcommand{\p}{\hphantom{+}}
\newcommand{\vect}[1]{\mbox{\boldmath $ #1 $}}
\newcommand{\reff}[2]{\ref{#1}.\ref{#2}}
\newcommand{\psum}[2]{\sum_{#1}^{#2}\!\!\!'\,\,}
\newcommand{\bin}[2]{\left( \begin{array}{@{}c@{}}
				#1 \\ #2
			\end{array}\right)	}
%
%  Macros - some of these are in plain TeX (gasp!)
%
\newcommand{\be}{($\beta$)}
\newcommand{\eqp}{\mathrel{{=}_p}}
\newcommand{\ltp}{\mathrel{{\prec}_p}}
\newcommand{\lep}{\mathrel{{\preceq}_p}}
\def\brack#1{\left \{ #1 \right \}}
\def\bul{$\bullet$\ }
\def\cl{{\rm cl}}
\let\del=\partial
\def\enditem{\par\smallskip\noindent}
\def\implies{\Rightarrow}
\def\inpr#1,#2{\t \hbox{\langle #1 , #2 \rangle} \t}
\def\ip<#1,#2>{\langle #1,#2 \rangle}
\def\lp{\ell^p}
\def\maxb#1{\max \brack{#1}}
\def\minb#1{\min \brack{#1}}
\def\mod#1{\left \vert #1 \right \vert}
\def\norm#1{\left \Vert #1 \right \Vert}
\def\paren(#1){\left( #1 \right)}
\def\qed{\hfill \hbox{$\Box$} \smallskip}
\def\sbrack#1{\Bigl \{ #1 \Bigr \} }
\def\ssbrack#1{ \{ #1 \} }
\def\smod#1{\Bigl \vert #1 \Bigr \vert}
\def\smmod#1{\bigl \vert #1 \bigr \vert}
\def\ssmod#1{\vert #1 \vert}
\def\sspmod#1{\vert\, #1 \, \vert}
\def\snorm#1{\Bigl \Vert #1 \Bigr \Vert}
\def\ssnorm#1{\Vert #1 \Vert}
\def\sparen(#1){\Bigl ( #1 \Bigr )}

\newcommand\blankpage{%
    \null
    \thispagestyle{empty}%
    \addtocounter{page}{-1}%
    \newpage}

%%%%%%%%%%%%%%%%%%%%%%%%%%%%%%%
%
% These environments allow you to get nice numbered headings
%  for your Theorems, Definitions etc.  
%
%  Environments
%
%%%%%%%%%%%%%%%%%%%%%%%%%%%%%%%

\newtheorem{theorem}{Theorem}[section]
\newtheorem{lemma}[theorem]{Lemma}
\newtheorem{proposition}[theorem]{Proposition}
\newtheorem{corollary}[theorem]{Corollary}
\newtheorem{conjecture}[theorem]{Conjecture}
\newtheorem{definition}[theorem]{Definition}
\newtheorem{example}[theorem]{Example}
\newtheorem{remark}[theorem]{Remark}
\newtheorem{question}[theorem]{Question}
\newtheorem{notation}[theorem]{Notation}
\numberwithin{equation}{section}

%%%%%%%%%%%%%%%%%%%%%%%%%%%%%%%%%%%%%%%%%%%%%%%%%%%%%%%%%%%%%%%%%%
%
%  If you've got some funny special words that LaTeX might not
% hyphenate properly, you can give it a helping hand:
%

\hyphenation{Mar-cin-kie-wicz Rade-macher}






\begin{document}

\beforepreface

\prefacesection{Abstract}

A first assessed activity will be the Group Project Plan (20\%). The
first step is to delimit the problem which will be studied. You should
understand and define your own overall problem and propose a solution.
Having only a short time to complete the project, it is crucial that the
problem that will be studied is well defined. The approach to solve the
problem should be original, so it will be necessary to carry out a
preliminary literature review. This will prevent any plagiarism. This
will also enable you to situate the project in a more global context. At
this stage, one can identify potential approaches and software that will
be used to solve the problem. It is necessary to plan any simulation
carefully and to decide what statistical analyses will have to be
carried out. All the sub-steps of the project should be planned
precisely (a detailed schedule will be created). A kind of short draft
version of your final report, consisting of a three-page proposal
(excluding the mandatory pages devoted to the title, abstract, contents
and references) will be submitted and discussed with one of the
instructors at the end of Week 1 or in a clear definition of the problem
you plan to study; a clear description of the data format and their
storage; a clear description of the relevant data (variables, missing
and corrupt values, etc.); the level of difficulty of the chosen data
sets (size, complexity, messiness) and its relevance for the chosen
problem; the appropriate choice of software and statistical methods to
solve your research questions; a clear description of the role of each
team member with a proper justification; the precision of your scheduled
activities.

%\afterpage{\blankpage}


\afterpreface





%%%%%%%%%%%%%%%%%%%%%%%%%%%%%%%%%%%%%%%%%%%%%%%%%%%%%%%%%%%%%%%%%%
%
% Now we can start on the first chapter
% Within chapters we have sections, subsections and so forth
%
%%%%%%%%%%%%%%%%%%%%%%%%%%%%%%%%%%%%%%%%%%%%%%%%%%%%%%%%%%%%%%%%%%



%%%%%%%%%%%%%%%%%%%%%%%%%%%%%%%%%%%%%

%\afterpage{\blankpage}


\setcounter{chapter}{1}
\renewcommand\thesection{\arabic{section}}

\hypertarget{introduction-and-motivation}{%
\section{Introduction and
Motivation}\label{introduction-and-motivation}}

The Covid-19 pandemic resulted in a myriad of changes in the behaviour
of individuals, businesses, and global economy. In the context of energy
forecasting, the lockdowns due to the pandemic and the advent of working
from home are interesting phenomena that have left notable impressions on the
demand and supply of energy. Additionally, the growth of the digital economy
was exacerbated as a result of the events of the pandemic and this in itself
has transformed consumer and business energy use.

Recent studies on global and Australian energy demand have indicated:
\begin{enumerate}
  \item Overall demand has reduced whilst residential demand has increased
  \item The pattern of demand has changed during the day and during the week
\end{enumerate}

Our goal in this analysis is to isolate the effects of lockdowns, and working
from home from the pre-Covid trend in the present energy demand curve. In doing
so we can generate valuable insights on how the persistent shift to working from
home will continue to effect energy demand in the future.

\bigskip

To achieve this goal we will begin by determining a high-accuracy pre-Covid 5 minute
energy demand forecasting model. To obtain this we will compare Multiple Linear
Regression models, Random Forests, and Neural Networks using the weather and total
energy demand data.

This model will be applied to the Covid-to-present portion of the data and the output
will be compared with the actual values to identify the impression of the pandemic
and working from home on the pre-Covid trend.

\hypertarget{brief-literature-review}{%
\section{Brief Literature Review}\label{brief-literature-review}}

Survey the most related work you have found in the literature. Due to
space restrictions, describe only the most relevant work and discuss its
connections to your work. Describe the methods employed in the related
work, as well as the employed measure of success.

\bigskip

In order to incorporate your own references in this report, we strongly
advise you use BibTeX. Your references then need to be recorded in the
file \texttt{references.bib}, and cited as follows {[}1{]}.

\hypertarget{methods-software-and-data-description}{%
\section{Methods, Software and Data
Description}\label{methods-software-and-data-description}}

You should describe the techniques you intend to use and why you have
selected these techniques. Also, describe the software and libraries
that you will need to implement your analyses. You can revise this later
since this is an initial proposal. Finally, provide a clear description
of the relevant data (format, storage, variables, messiness, size,
complexity) and its relevance for the chosen problem.

\hypertarget{activities-and-schedule}{%
\section{Activities and Schedule}\label{activities-and-schedule}}

List the main project activities and create a timetable for the
activities. You can use a Gantt Chart (see next page). Describe the
roles of all team members with proper justification.

\newpage

Here is an example of a Gantt chart created using \LaTeX:

\bigskip

\hypertarget{references}{%
\chapter*{References}\label{references}}
\addcontentsline{toc}{chapter}{References}

\bibliographystyle{elsarticle-num}
\bibliography{references}

\hypertarget{refs}{}
\begin{CSLReferences}{0}{0}
\leavevmode\vadjust pre{\hypertarget{ref-Lafaye2013}{}}%
1. Lafaye de Micheaux P, Drouilhet R, Liquet B. The r software:
Fundamentals of programming and statistical analysis. Springer New York;
2013.

\end{CSLReferences}







\end{document}

